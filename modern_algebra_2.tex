\documentclass[11pt]{kbook}

\usepackage{math}

%TODO change this to book

%---MATH MACROS---

\addbibresource{bibliography.bib}

%---TITLE---

\title{Lecture Notes to a Course on Non-Commutative Algebra \\ \small{Taught by Prof. Eli Aljadeff at Technion IIT during Spring 2022}}
\author{Typed by Elad Tzorani}
\date{\today}

\DeclareMathOperator{\Br}{Br}
\DeclareMathOperator{\Mor}{Mor}
\newcommand{\op}{{\mrm{op}}}

\begin{document}

\maketitle

%TODO edit beginning sections and add course information

\chapter{Introduction}

\section*{Course Information}

%TODO fill in

\section*{Prerequisites}

The course will assume undergraduate knowledge in group theory and ring theory.

\section*{Notations \& Conventions}

%TODO organize this

\begin{itemize}
\item All rings and $k$-algebras are assumed to be associative and unital, but not necessarily commutative.
\end{itemize}

\section*{Course Goals \& Motivation}

\subsection*{Course Goals}

During the course we will go over the following topics.

\begin{itemize}
\item Category Theory \& other algebraic tools.

\item Rings \& the Jacobson radical.

\item The Wedderburn-Artin theorem.

\item Central simple algebras.

\item Brauer group theory.

\item Group cohomology \& Galois cohomology.

\end{itemize}

\subsection*{Motivation}

Let $k$ be a field and consider finite-dimensional $k$-algebras $A$ that are \emph{simple} in the sense that there are no nontrivial two-sided ideals. Assume furthermore that $A$ is \emph{central} in the sense that its centre, $Z\prs{A}$, is $k$.
Wedderburn-Artin theorem, any such algebra is a matrix algebra over division algebras.

\begin{example}
The following are central simple algebras over a field $K$:

\begin{enumerate}
\item $K$.
\item $M_n\prs{K}$ for any $n \in \mbb{N}$.
\item \[\mbb{H}_K \coloneqq K\trs{i,j,k}{\substack{i^2 = j^2 = k^2 \\ ij = ji = k \\ ik = -ki = -j \\ jk = -kj = i}} \text{.}\]
Over $\mbb{C}$ we have $\mbb{H}_{\mbb{C}} \cong M_2\prs{\mbb{C}} \subseteq \mbb{H}_{\mbb{R}}$.
\end{enumerate}
\end{example}

\begin{example}
There are no finite-dimensional division algebras over an algebraically closed field $k$.

Let $k$ be an algebraically closed field, and let $D$ be a finite-dimensional $k$-division algebra. For $z \in D$ one has that $k\prs{z}$ is a field, which is finite over $k$. Since $k = \bar{k}$ it follows that $k\prs{z} = k$, so $z \in k \cdot 1$. Hence $D = 1 \cdot k \cong k$.
\end{example}

\begin{theorem}[Wedderburn-Artin]
Let $k$ be a field and $A$ a central simple algebra over $k$. There is a $k$-division algebra $D$ and $r \in \mbb{Z}_+$ such that $A \cong M_r\prs{D}$. Furthermore, $r$ is uniquely determined, and $D$ is uniquely determined up to a $k$-algebra isomorphism.
\end{theorem}

\begin{definition}[Brauer Equivalence]
Let $A,B$ be central simple algebras over a field $k$. Write $A \cong M_{r_1}\prs{D_1}$ and $B \cong M_{r_2}\prs{D_2}$ as in Wedderburn-Artin. We say that $A,B$ are \emph{Brauer equivalent}, and write $A \sim B$, if $D_1 \cong D_2$.

The Brauer (equivalence) class of a central-simple algebra $A$ is denoted $\brs{A}$. We denote the set of Brauer equivalence classes of central simple algebras over $k$ by $\Br\prs{k}$. This directly corresponds to $k$-central division algebras. This object is helpful because there is a natural group structure on it.
\end{definition}

\begin{example}
For every field $k$, one has a Brauer equivalence $M_r\prs{k} \sim k$.
\end{example}

\begin{proposition}\label{proposition:CSA_product}
Let $A,B$ be central simple algebras over a field $k$. Then their \href{https://en.wikipedia.org/wiki/Tensor_product_of_algebras}{tensor product} $A \otimes_k B$ is also a central simple algebra.
\end{proposition}

In light of \Cref{proposition:CSA_product} we can define the following product on $\Br\prs{k}$.

\begin{definition}[Brauer Group]
Let $k$ be a field and let $\brs{A},\brs{B} \in \Br\prs{k}$. We define the product
\[\brs{A} \brs{B} \coloneqq \brs{A \otimes_k B} \text{.}\]

This has the unit $k$ and $\brs{A}$ has an inverse $\brs{A}^{-1} = \brs{A^\op}$ where $A^\op$ is the \emph{opposite algebra} which is $A$ as a set, with product $a \cdot_\op b = b \cdot a$. This turns $\Br\prs{k}$ into a group.
\end{definition}

\begin{proposition}
If $A$ is a central simple algebra over $k$ of dimension $m^2$, then $A^\op$ is of dimension $m$, then $A \otimes_k A^\op \cong M_{m^2}\prs{k} \sim k$.
\end{proposition}

\begin{example}
Let $k = \mbb{R}$. There are only two division algebras over $k$, which are ${\mbb{R}, \mbb{H}_{\mbb{R}}}$. Then $\Br\prs{\mbb{R}} \cong \mbb{Z}/2\mbb{Z}$. We get that $\mbb{H}_{\mbb{R}} \otimes_{\mbb{R}} \mbb{H}_{\mbb{R}} \cong M_4\prs{\mbb{R}}$.
\end{example}

\chapter{Preliminaries}

\section{Some Category Theory}

\subsection{Basic Definitions}

A central study in algebra is that of category theory. Through this we can study collections of objects such as those of all abelian groups, all topological spaces, et cetera.
There are obviously set-theoretic issues and one has to define things correctly. We will ignore them completely. For us, any collection of sets is called a \emph{class} and we will work with classes without elaboration of set-theoretic issues.

\begin{definition}[Category]
A \emph{category} is $\mcal{C}$ is a triple $\prs{\ob\prs{\mcal{C}}, \Mor\prs{\mcal{C}}, \circ}$ that satisfies the following.

\begin{enumerate}[label = (\roman*)]
\item $\ob\prs{\mcal{C}}$ is a class of what we call \emph{objects}.
\item For all $A,B \in \ob\prs{\mcal{C}}$ there is a set $\hom_{\mcal{C}}\prs{A,B}$.
\item For all $A,B,C \in \ob\prs{\mcal{C}}$ there is a \emph{composition}
\[\circ \colon \hom_{\mcal{C}}\prs{A,B} \times \hom_{\mcal{C}}\prs{B,C} \to \hom_{\mcal{C}}\prs{A,C}\]
subject to the following conditions.
\begin{enumerate}[label = (\Roman*)]
\item For every $A \in \ob\prs{\mcal{C}}$ there is an element $1_A \in \End_{\mcal{C}}\prs{A} \coloneqq \hom_{\mcal{C}}\prs{A,A}$ such that $1_A \circ f = f$ and $g \circ 1_A$ for all $f,g$ (such that composition makes sense).
\item Composition is associative.
\end{enumerate}
\end{enumerate}
\end{definition}

\begin{example}[$\mathbf{Set}$]
The collection $\mathbf{Set}$ of sets and functions between them is a category. Composition is that of functions.
\end{example}

\begin{example}[$\mathbf{Group}$]
The collection $\mathbf{Group}$ of groups and group homomorphisms is a category. Composition is that of homomorphisms.
\end{example}

\begin{example}[$\mathbf{Ab}$]
The collection $\mathbf{Ab}$ of abelian groups and their homomorphisms is a category. Composition is that of homomorphisms.
\end{example}

\begin{example}[$\mathbf{Top}$]
The collection $\mathbf{Top}$ of topological spaces with continuous functions between them is a category. Composition is that of continuous functions.
\end{example}

\begin{example}
Let $X$ be a non-empty set with a reflexive and transitive relation $R$. We define a category $\mcal{C}$ with $\ob\prs{\mcal{C}} = X$ and where
\[\hom_{\mcal{C}}\prs{a,b} = \begin{cases}\emptyset & \prs{a,b} \notin R \\ \set{*} & \prs{a,b} \in R\end{cases}\]
and composition is the only one possible.
\end{example}

\begin{example}[$\mathbf{Monoid}$]
Let $\mathbf{Monoid}$ be the category of \href{https://en.wikipedia.org/wiki/Monoid}{monoids} together with their homomorphisms.
\end{example}

\begin{example}
Let $R$ be a (unital) ring. Then $\prs{R, \times, 1}$ is a monoid. We define a category $\mbf{B}\prs{R}$ with one object $*$ and where $\hom_{\mbf{B}\prs{R}}\prs{*, *}$ and composition is defined by the multiplicative structure.
\end{example}

\begin{example}
Let $\mbf{Set}^*$ be the category of pairs $\prs{A,p}$ where $A$ is a set and where $p \in A$, and where morphisms $\prs{A,p} \to \prs{B,q}$ are functions $f \colon A \to B$ with $f\prs{p} = q$.
\end{example}

\begin{example}
\begin{enumerate}
\item Let $R = k$ be a field. An $R$-module is a $k$-vector space.

\item Let $R = \mbb{Z}$. An $R$-module is an abelian group $A$ with the action
\[ m * a = ma \coloneqq \overset{\text{$m$ times}}{\overbrace{a + \ldots + a}} \text{.} \]
\end{enumerate}
\end{example}

\begin{example}
Let $k$ be a field. We have $k^n$ and $M_{n, r}\prs{k}$ as $M_n\prs{k}$-right modules. We later show that these are all the possibilities, up to isomorphism.
\end{example}

\begin{notation}
We denote by $R\mbf{-Mod}$ the category of all left $R$-modules with their morphisms.
\end{notation}

\begin{definition}[Isomorphism]
Let $\mcal{C}$ be a category. An isomorphism $f \colon X \to Y$ is a morphism that admits an inverse $g \colon Y \to X$ (i.e. such that $g \circ f = \id_X$ and $f \circ g = \id_Y$).
\end{definition}

\section{Some Ring Theory}

Let $\pi \colon R \to S$ be a ring homomorphism. Then $R$ can be considered an $S$ module via $s \cdot r \coloneqq \pi\prs{s} r$.
Similarly, any $R$-module $M$ can be considered an $S$-module via $s \cdot m \coloneqq \pi\prs{s} m$. For a ring $R$ and a group $G$ we define $RG$ to be the free $R$-algebra generated by elements of $G$. For a set $X$ we denote by $R_X$ the free $R$-module generated by $X$.

By looking at the definitions, we see that a $kG$-module is exactly a $k$-vector space with a $G$-action. This allows the study of representations (maps from $G$ to $\mrm{GL}_d\prs{k}$) of $G$ via the study of $kG$.

\begin{theorem}
Let $R$ be a commutative ring and assume that $R^m \cong R^n$. Then $m = n$.
\end{theorem}

We prove the above theorem later in the course using tensor products.

\begin{proposition}
Let $R$ be a commutative ring, let $F_1 = R X_1$ and let $F_2 = R X_2$ where $X_i$ are sets such that $\abs{X_1} = m$ and $\abs{X_2} = n$.
Then $F_1 \cong F_2$ as $R$-modules.
\end{proposition}

\begin{notation}
Let $R$ be a ring and $M$ an $R$-module. For $r \in R$ we define
\begin{align*}
L_r \colon M &\to M \\
m &\mapsto rm \text{.}
\end{align*}
\end{notation}

\begin{remark}
In the above notation, $L_r$ is necessarily a $\mbb{Z}$-module homomorphism. However,
\[L_r\prs{sm} = rsm, \quad s L_r\prs{m} = srm\]
and these aren't generally equal. Hence, $L_R$ isn't necessarily an $R$-module homomorphism.

Still, we can find something stronger than being a map of $\mbb{Z}$-modules.
If $s \in Z\prs{R}$, we have $rsm = srm$, and it follows that $L_r$ is a ${Z}\prs{R}$-module homomorphism. Similarly, if $r \in Z\prs{R}$, $L_r$ is necessarily an $R$-module homomorphism.
\end{remark}

\begin{definition}
Let $R$ be a ring. We define $R^\op$ to be the ring with the same elements but reverse multiplication.
\end{definition}

\begin{remark}
A left $R$-module is a right $R^\op$-module via
\[m \cdot r^\op = r m \text{.}\]
\end{remark}

%4.4.22

We notice that in the category of (left) $R$-modules, for any objects $A,B$ the set $\hom_R\prs{A,B}$ has an $R$-module structure.
For $r \in R$ and $\hom_R\prs{A,B}$ we might want to define $\prs{rf}\prs{a} = r \cdot f\prs{a}$, but this isn't necessarily a map of $R$-modules: we would need to have $\prs{rf}\prs{ta} = t \prs{rf}\prs{a}$, but $r t f\prs{a}, t r f\prs{a}$ aren't generally equal when $R$ is noncommutative.

\begin{example}
Let $R,S$ be rings and $A$ is an $R-S$ bimodule (i.e. a left $R$-module and a right $S$-module, such that the actions respect each other).
Defining
\begin{align*}
sf \colon A &\to B \\
a &\mapsto f\prs{as}
\end{align*}
we have
\begin{align*}
\prs{sf}\prs{ra} &= f\prs{\prs{ra}s}
\\&= f\prs{r\prs{as}}
\\&= rf\prs{as}
\\&= r\prs{sf}\prs{a} \text{,}
\end{align*}
hence $sf$ is an $R$-module homomorphism.
One similarly check that $\prs{s_1 s_2} f = s_1 \prs{s_2 f}$.
\end{example}

We now ask when $\hom_R\prs{A, -}$ is additive (i.e. the hom sets are homomorphisms of abelian groups). Taking a map $f \colon B \to B'$ we get
\begin{align*}
f_* \colon \hom_R\prs{A,B} &\to \hom_R\prs{A,B'} \\
g &\mapsto f \circ g \text{.}
\end{align*}
Then $\prs{f_1 + f_2}_* = \prs{f_1}_* + \prs{f_2}_*$, so $\hom_R\prs{A,-}$ is indeed additive.

\begin{example}
Let $R$ be a commutative ring. A left $R$-module $M$ is then also a right $R$-module with $m \cdot r \coloneqq r \cdot m$. It is easily checked that $M$ is then an $R-R$-bimodule.
\end{example}
%---BIBLIOGRAPHY & INDEX---

\printbibliography

\end{document}